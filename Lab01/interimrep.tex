\documentclass{article}
\usepackage[margin=1in]{geometry}
\usepackage{amsmath}
\usepackage{graphicx}

\newcommand{\quickfig}[2]{
	\begin{figure}[th]
		\centering
		\includegraphics[width=0.8\textwidth]{InterimPics/figure_#1.pdf}
		\caption{\label{fig:f#1}#2}
	\end{figure}
}

\title{Astronomy 120 Optical Lab \\ Lab 1 Interim Report}
\author{Ramsey Karim {\small \sl with} Orders of Raditude Research Group}

\begin{document}

\maketitle
\abstract{
	These are the first 8 plots pertaining to Lab 1.
	They involve some basic plotting and statistics using the
	photon count data from the CoinPro software.
	Each plot uses an arbitrary set of data, but all use
	total $N = 10^4$.
	The ``OFF'' and ``MAX'' runs were never used for this report.
}


\section{All Figures (1 - 8)}
\quickfig{1}{Time series of clock tick by event number. Note the cyclical nature of the time stamps
	 -- this is due to the time stamp data type (int 32), which has upper and lower limits and
	 deals with overflow using modular arithmetic based on those limits.}
\quickfig{2}{Time series of interval by event number. Random variation is apparent but difficult
	to see due to the linestyle of the plot.}
\quickfig{3}{Time series of interval by event number -- same as Figure \ref{fig:f2}, but with the lines suppressed,
	making the nature of the variation (and clustering around $dt = 0$ more apparent.)}
\quickfig{4}{The mean interval from Figure \ref{fig:f3}, calculated in chunks of 1000 elements. Note the variation, but the
	general clustering around a range of values (the $y$ axis range).}
\quickfig{5}{The mean interval from Figure \ref{fig:f3}, calculated in progressively larger chunks all starting
	from the first element. Note the convergence to the mean of the entire set, indicated by the final value.}
\quickfig{6}{The mean interval from Figure \ref{fig:f3}, calculated in chunks of 100 elements. This is the same as
	Figure \ref{fig:f4} but with smaller subsets of data. Standard deviation of the mean increases when subset
	size decreases, as one can see from the increase in variation here compared to Figure \ref{fig:f4}.}
\quickfig{7}{Standard deviation of the mean (SDOM) plotted against the subset size used to calculate the SDOM.
	Note the clear correlation between large subset size and small SDOM.}
\quickfig{8}{SDOM versus $1/\sqrt{N}$. Note that the data points heavily suggest a linear correlation to this value.
	They very well match the theoretical curve of SDOM $= \frac{\sigma}{\sqrt{N}}$ (the plotted line), where $\sigma$ is the standard
	deviation of the entire set (ideally $\lim_{N \to \infty}$, but $N = 10^4$ here, since $10^4 \gg$ the subset sizes
	used for the plotted points).}


\end{document}